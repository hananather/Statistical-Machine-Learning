\section{January 11, 2023}
\subsection{OLS as ERM}
\textbf{Regression} is class of statistical/computational methods that seek to estimate relationships between a dependent (target) variable and independent variables (instances/features/inputs).
There are many different techniques used for regression: linear regression, support vector machines, neural networks, K-nearest neighbours, random forest.
The differences among the methods boils down to two factors: how we choose represent the function/hypothesis $f: X \to Y$ and what \textit{loss function} we choose to measure the 'goodness' of our model. 
In linear regression we assume the $f$ function we are seeking belongs to the class of linear functions, which we denoted as:  $\SF = \{\bw \cdot \bx | w \in \RR^d \}$ or in a expanded form:
$$
f(x) = w_0 + w_1 x_1 + \cdots + w_d x_d
$$
Now given a data set of input variables $\bx$ and labels $y$, how can we reasonably find the weights $\bw$? One reasonable approach is to find $\bw$ such that $\bw \cdot \bx$ is close to $y$.
There are many different loss functions to find the weight vector $w \in \RR^d$
The \textbf{least squares or L2 loss} is $V(f,z) = (y - f(x))^2$. In this setting  we let $X = \RR^d$ and $Y \subset \RR$. 
\subsection{Mathematical formalization of linear regression}
We store the data into two matrices: $y$ is an $m \times 1$ column vector containing \textit{labels}, and $X$ is a $m \times d$ matrix containing \textit{inputs} or \textit{features}. We define an \textit{objective function} on $\RR^d$

\begin{equation}\label{OLS objective}
    J(w) = \parallel y - w\cdot x \parallel^2
\end{equation}
The $w$ which minimizes (\ref{OLS objective}) $\Leftrightarrow$ solves $Aw = b$ where $b = X^Ty$ and $A = X^TX$. 
Let $\SM = \{w|Aw =b \}$ be the set of minimizers (\ref{OLS objective}) and let 
$\SN =$ ker $A \subset \RR^d$. Since $A = X^TX$ is a symmetric matrix by construction, we know from linear algebra that $\SN \perp$ Col $A$. Therefore,  $b = X^Ty \in $ Col$(A)$, which implies there exists at least one minimizer to (\ref{OLS objective}) or equivalently, $\SM \neq \varnothing$
\footnote{When $A = X^TX$ is \textit{invertible} there is a unique solution, otherwise, there are infinitely many solutions}.



\subsection{Linear Algebra facts}
Properties of Gram matrix:
\begin{itemize}
    \item $A = X^T X$ is called the \vocab{gram matrix}
    \item $A$ is a \textit{symmetric} matrix (i.e., $A^T = A$). This implies that null$(A)$ $\perp$ Col$(A)$ and there exists an orthonormal basis.
    \item $A = \sum_{i}x_ix_{i}^T$ is the sum of rank $1$-matrices.
    \item $A$ is \textbf{semi-definite} ($u^TAu \geq 0$)
    \item Nullspace of $A$ coincides with null space of $X$; Null(X) = Null(A) = $\{v \in \RR^d : Av =0\}$
    \item $X^T X$ is invertible $\iff$ $X$ is invertible $\iff$ $X$ has a null space of dimension zero $\iff$ all columns of $X$ are independent
    \item The \textit{inverse} of non-square matrices is not defined. The \textbf{left inverse} is given by $X^{\dag} := (X^TX)^{-1}X^T $
\end{itemize}
When solving the equation $Aw =b$ there are three possibilities: (1) There exists a unique solution if $A$ is invertible (2) infinitely many solutions if $b \perp$ null($A$) (3) No solutions when $b \not\perp$ null($A$). Since in the OLS case $b = X^Ty$ the condition $b \perp$ null($A$) is always satisfied.\footnote{Thus we always have at least one solution to OLS}
\subsection{Spectral decomposition}
\begin{itemize}
    \item We have \textit{basis of eigenvectors} $u_1, u_2, \dots$ for $A$; 
    real eigenvalues $\lambda_i \geq 0$, indexed in ascending order
    \item $\SN$ is span span of $u_i$ with $\lambda_i = 0$, let $r =$dim($\SN$).
    \item In terms of basis $\{u_i\}$, write $w = \sum \alpha_i u_i$ and $b = \beta_i u_i$.
    \item Matrix equation $Aw = b$ can be re-written in terms of eigenvalues and eigen vectors
        $$
        \sum_{i > r} \alpha_i \lambda_i u_i = \sum_{i > 0} \beta_i u_i 
        $$
    \item there exists a unique solution if and only if $\beta_1, \dots, \beta_r = 0$ 
    (i.e., $b \perp \SN$) \footnote{\term{why? please look into this and explain this in the notes}}

    \item And $w = \sum \alpha_i u_i$ if and only if $\alpha_i = \frac{\beta_i}{\lambda_i}, \forall i > r$
    
    \item $\alpha_1, \dots, \alpha_r$ are arbitrary due to the fact we that we have a degree of freedom for every $\lambda_1, \dots, \lambda_r = 0$.
\end{itemize}

Since A is a \textit{real symmetric} matrix there exits an orthonormal basis $\{u_1, \cdots, u_d\}$ for $\RR^d$ consisting of egienvectors.\footnote{This is known as the spectral theorem for symmetric matrices}

 